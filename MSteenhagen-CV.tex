%%% A template to produce a nice-looking Curriculum Vitae.
%%% Orginally produced by Kieran Healy <kjhealy@gmail.com>
%%% This version is adapted (only minimally) by Maarten Steenhagen.
%%% 
%%% ------------------------------------------------------------------------
%%% Requirements (should be included in a modern tex distribution):
%%% ------------------------------------------------------------------------
%%% xelatex
%%% fontspec.sty
%%% hyperrref.sty
%%% xunicode.sty
%%% color.sty
%%% url.sty
%%% fancyhdr.sty
%%%
%%% ------------------------------------------------------------------------
%%% Optional
%%% ------------------------------------------------------------------------
%%% git
%%% vc.sty
%%% revnum.sty
%%% Fonts
%%%
%%% ------------------------------------------------------------------------
%%% Note
%%%------------------------------------------------------------------------
%%% Because this is a hand-tweaked file, be on the look out for \medksip, 
%%% \bigskip and \newpage commands here and there, which are used to balance
%%% the layout or avoid widows & orphans, etc. You should of course add or 
%%% remove these as needed.
%%%------------------------------------------------------------------------

\documentclass[12pt]{article}

%%%------------------------------------------------------------------------
%%% Metadata
%%%------------------------------------------------------------------------

%% Change as needed. Or just add me as a coauthor. Only some of these are 
%% used below in the hyperref declaration and address banner section.
\def\myauthor{Maarten Steenhagen}
\def\mytitle{Vita}
\def\mycopyright{\myauthor}
\def\mykeywords{}
\def\mybibliostyle{plain}
\def\mybibliocommand{}
\def\mysubtitle{}
\def\myaffiliation{University of Antwerp}
\def\myaddress{Centre for Philosophical Psychology}
\def\myemail{maartensteenhagen@gmail.com}
\def\myweb{http://msteenhagen.github.io}
\def\myblog{http://maarten.steenhagen.nl}
\def\myphone{+44 (0)78 66 425161}
\def\myfax{~}
\def\myversion{}
\def\myrevision{}


% \def\myaffiliation{University of Antwerp}
\def\myauthor{Maarten Steenhagen}
\date{} % not used (revision control instead)
\def\mykeywords{Maarten, Steenhagen, Maarten Steenhagen, Vita, CV, Resume, Philosophy}

%%%------------------------------------------------------------------------  
%%% Git version tracking 
%%%------------------------------------------------------------------------

%% If you don't use git or the vc package (from CTAN), comment this out.
%% If you comment it out, be sure to remove the \rfoot comment below, too.
%\input{vc}

%%%------------------------------------------------------------------------
%%% Required style files
%%%------------------------------------------------------------------------
\usepackage{url,fancyhdr}
%%\usepackage{revnum} % for reverse-numbered publications (revnumerate environment) if needed.

%% needed for xelatex to work
\usepackage{fontspec}
\usepackage{xunicode}
\usepackage{multicol}
%% color for the links 
\usepackage{color}
\definecolor{grey}{cmyk}{0,1,0,0.8}
%% hyperlinks
\usepackage[xetex, 
	colorlinks=true,
	urlcolor=grey,
	plainpages=false,
  	pdfpagelabels,
  	bookmarksnumbered,
  	pdftitle={\mytitle},
  	pagebackref,
  	pdfauthor={\myauthor},
  	pdfkeywords={\mykeywords}
  	]{hyperref}

%%%------------------------------------------------------------------------
%%% Document
%%%------------------------------------------------------------------------
\begin{document}

%% Choose fonts for use with xelatex
%% Minion and Myriad are widely available, from Adobe. 
%% Pragmata is available to buy at http://www.fsd.it/fonts/pragma.htm
%% and is worth every penny. Any good monospace font will work fine, though.
%% Consolas or inconsolata are good alternatives.
\setromanfont[Mapping={tex-text},Numbers={OldStyle},Ligatures={Common}]{Brill} 
\newfontfamily\margin[Mapping=tex-text,Colour=AA0000, Letters=SmallCaps]{Futura}
\setmonofont[Mapping=tex-text,Scale=0.9]{Helvetica} 

% \newcommand{\amps}{\amper{\&}}

%%%------------------------------------------------------------------------
%%% Local commands
%%%------------------------------------------------------------------------

%% Marginal header
%% Note: as the document goes on you may need to introduce a (gradually increasing)
%% \vspace element to keep the marginal header pleasingly aligned with the first 
%% item in the body text. Like this: \marginhead{{\vskip 0.4em}Grants}, or 
%% \marginhead{{\vskip 0.8em}Service}. Experiment as needed.
\newcommand{\marginhead}[1]{\marginpar{\margin{{\footnotesize\vspace{-1em}\flushright #1}}}}


%% custom ampersand (font consistent with the one chosen above)
\newcommand{\amper}{{\fontspec[Scale=.85]{Baskerville Italic}\selectfont\&\,}}

%% No bullets on labels
\renewcommand{\labelitemi}{~} 

%% Custom hanging indent for vita items
\def\ind{\hangindent=1 true cm\hangafter=1 \noindent}
%\def\ind{\hangindent=18pt\hangafter=1 \noindent}
\def\labelitemi{~}
\renewcommand{\labelitemii}{~}

%%%------------------------------------------------------------------------
%%% Page layout
%%%------------------------------------------------------------------------
\pagestyle{fancy}
\renewcommand{\headrulewidth}{0pt}
\fancyhead{}
\fancyfoot{}
\rhead{{\scriptsize\thepage}}

%% git revision control footer 
%\rfoot{\texttt{\scriptsize \VCRevision\ on \VCDateTEX}} % git revision info inserted via external script -- see docs for vc package for details. comment out this line if you're not using vc, and also remove the \input{vc} line above.

%%%------------------------------------------------------------------------
%%% Address and contact block
%%%------------------------------------------------------------------------
\begin{minipage}[t]{2.95in}
 \flushright {\footnotesize {Centre for Philosophical Psychology} \\ University of  Antwerp \\ Stadscampus \\ Lange Sint-Annastraat 7, S.S.208 \\ \vspace{-0.05in} 2000 Antwerp, Belgium}  
  
\end{minipage}
\hfill     
%\begin{minipage}[t]{0.0in}
% dummy (needed here)
%\end{minipage}
\hfill
\begin{minipage}[t]{1.7in}
  \flushright \footnotesize Phone: \myphone \\  
  {\scriptsize  \texttt{\href{mailto:\myemail}{\myemail}}} \\
  {\scriptsize  \texttt{\href{\myweb}{\myweb}}}\\
	{\scriptsize  skype: \texttt{maartensteenhagen}}
\end{minipage}



\bigskip  
%% Name 
\noindent{\Large {\textsc{maarten steenhagen}}}
\reversemarginpar

\bigskip       

%% Appointments
\medskip

\marginhead{Current Position}

\ind Research Fellow\\
Centre for Philosophical Psychology\\
University of Antwerp

\vspace{0.01in}

\medskip

\marginhead{AoS}

\ind Philosophy of perception; Philosophy of psychology; Aesthetics

\bigskip
\bigskip

\marginhead{AoC}

\ind Metaphysics; History of analytic philosophy; Philosophy of language

% \ind Supervisor: Prof. Mark Eli Kalderon
% 
% \ind Expected: Spring 2015
% 
% \bigskip
\bigskip
\bigskip

\marginhead{{\vskip 0.3em}Publications}
\medskip
% 
\ind (forthcoming) `Psycho-analysis, philosophy, and the pictorial arts' in: Lacewing, Gibbs, Steenhagen (eds.) \emph{The Oxford Handbook of Philosophy and Psychoanalysis}. Oxford University Press.

\ind (forthcoming) \emph{The Oxford Handbook of Philosophy and Psychoanalysis}. Oxford University Press. (Associate editor)

\ind  (forthcoming). `Review of Lambert Wiesing's \emph{Sehen Lassen},' in: \emph{British Journal of Aesthetics}. DOI: 10.1093/aesthj/ayu016 

\ind  (2010). `Explaining the ugly: disharmony and unrestrained cognition in Kant' in: \emph{Esthetica} (November).

\bigskip
\marginhead{Education}

% \ind 2012--Present. Doctoral Candidate (PhD).      

\ind 2014-2015. Visiting Research Student in the Department of Philosophy at the University of Toronto. Advisor: Mohan Matthen.

\ind 2011-2015. Ph.D., Philosophy. Dissertation:
\emph{Appearance and Representation}. 
\emph{University College London, Department of  Philosophy \vspace{0.01in}}\\ Advisors: Prof. Mark Eli Kalderon; Prof. Mohan Matthen (Toronto)\\
Examiners: Prof. John Hyman (Oxford); Prof. David Papineau (KCL) 
\vspace{-0.1in}

\begin{quote}
	\textsc{abstract:} In this dissertation I examine the relation between two concepts that are indispensable both to the philosophy of perception and to philosophical aesthetics: appearance and representation. Either implicitly or explicitly, philosophers have disagreed over whether representations play any significant role in our perception of objects, events or properties. In this work I show that imagistic representations, such as photographs, paintings, and drawings, indeed play a significant part in our perception of the items we encounter in the world. My main claim is that such images fulfil a mediating function in vision, allowing us to perceive the scenes they represent by means of them. Images enable us visually to perceive scenes not present to the sense of sight, and in this way introduce into the world the possibility for a distinctive kind of visual appearance.
\end{quote}

\ind 2011. MPhil Stud., Philosophy. Dissertation: \href{http://discovery.ucl.ac.uk/1348204/}{\emph{The Collective Image}}. \emph{University College London, Department of  Philosophy \vspace{0.01in}}\\ Supervisors: Prof. Mark Eli Kalderon; Prof. Christopher Peacocke \vspace{-0.1in}

\bigskip

\ind 2009. B.A., Philosophy. \emph{University of Utrecht, Department of  Philosophy\vspace{0.02in}}

\medskip

\ind 2005. BFA., Fine Art. \emph{ArtEZ School of Fine Art, Department of Monumental Art\vspace{0.02in}}

\bigskip

% \marginhead{{\vskip 0.3em}Publications}
% \medskip
% %
% \ind (forthcoming) \emph{The Oxford Handbook of Philosophy and Psychoanalysis}. Oxford University Press. (Associate editor)
%
% \ind  (forthcoming). `Review of Lambert Wiesing's \emph{Sehen Lassen},' in: \emph{British Journal of Aesthetics}. DOI: 10.1093/aesthj/ayu016
%
% \ind  (2010). `Explaining the ugly: disharmony and unrestrained cognition in Kant' in: \emph{Esthetica} (November).
% 
% \ind  Steenhagen, M. 2009. `Verwarring over lelijkheid,' \emph{Cimedart,} 39:2.
% 
% \ind  Steenhagen, M. 2008. `De keerzijde van liefde,' \emph{Cimedart,} 38:3.
% 
% \bigskip



\marginhead{{\vskip 0.4em}Scholarships \newline and awards}
\medskip

\ind 2015. Jacobsen Studentship, Royal Institute of Philosophy, London (£9000).

\ind 2014. American Philosophical Association, Postgraduate bursary (\$400).

\ind 2013. \href{http://www.ucl.ac.uk/ah/figs/homepage}{UCL Faculty of Arts and Humanities} Postgraduate Research Studentship (£5000) %(£36.000)

\ind 2011. \href{http://www.british-aesthetics.org}{British Society of Aesthetics} Two-year PhD Studentship Award (£36.000).

\ind 2011. Prins Bernhard Cultuurfonds (€10.000). \emph{Not taken up.}

\ind 2011. Hendrik Müller Vaderlandsch Fonds (€5.000). \emph{Not taken up.} 

\ind 2010. Dutch Association for Aesthetics Essay Contest, 2$^{\textrm{\tiny{nd}}}$ Prize.

\ind 2009. Huygens Scholarship Programme, Nuffic, The Hague (€30.000).

\bigskip 



%\newpage
%% Presentations
\marginhead{{\vskip 0.4em}Talks \newline(recent)}
\medskip

\ind 2016. ``False reflections'' American Philosophical Association, Eastern Division Meeting, San Diego. (Upcoming)

\ind 2015. ``Collingwood, sympathy, and the work of art'' INSEI workshop on empathy and interpretation. Cambridge, UK

\ind 2014. ``Source representationalism'' Postgraduate Sessions of the Joint Session of the Aristotelian Society and Mind Association, University of Cambridge.

\ind 2014. ``Thoughts you don't think'' American Philosophical Association, Pacific Division Meeting, San Diego. Respondent: Dr. Katherine Rickus (Marquette) 

% \ind 2013. ``Response to Renault--Steele'' American Society of Aesthetics annual conference, Department of Philosophy, University of California at San Diego. (Invited response.)

\ind 2013. ``Unthought thoughts'' Annual conference of the European Society for Philosophy and Psychology, University of Granada, Spain.

\ind 2013. ``Looking like: response to Sethi'' Berkeley-London Graduate Conference in Philosophy, Department of Philosophy, University of California at Berkeley. (Invited response.)

\ind 2013. ``Changing one's ways'' Workshop \emph{Skills, Habits, and Action Explanation}, Department of Philosophy, Birkbeck College London. (Response paper.)

\ind 2013. ``Pictures \& perceptual knowledge'' The Ockham Society, Department of Philosophy, University of Oxford.

\ind 2012. ``Two problems of co-consciousness.'' Department of Philosophy, University of Warwick, \href{http://www2.warwick.ac.uk/fac/soc/philosophy/news/conferences/mindgrad-2012/}{MindGrad 2012}. Respondent: Dr Christoph Hoerl (Warwick).

\ind 2012. ``Imaginary experience.'' University of Warwick, London-Warwick Graduate Mind Forum.

% \ind 2012. ``How the five die: Response to Lamb.'' University College London, 2012 Graduate Conference in Philosophy. (Response paper.)

\ind 2012. ```Non-conceptualised depiction.'' The Queen's College Oxford, Annual conference of the British Society of Aesthetics. September.

\ind 2012. ``Researching the self.'' St. John's College Oxford, \emph{Art as a Mode of Enquiry}.

\ind 2011. ``Seeing what another would.'' Utrecht University, \emph{Expert Meeting on Aesthetics}.

\ind 2011. ``Representing the trinity.'' University of Ghent, NGE Annual conference .

\ind 2011. ``Truly political? A psychoanalytic understanding of engaged art.'', University of Leeds, \emph{Working Through Psychoanalysis}.

% \ind 2010. ``Photography and dull causation: Response to Atencia-Linares.'' University College London, 2010 Graduate Conference in Philosophy. (Response paper.)

\ind 2010. ``Explaining the Ugly: Disharmony and Unrestrained Cognition in Kant.'' Institute of Philosophy, London-Berkeley Graduate Conference in Philosophy.

%\end{revnumerate}
\bigskip
\marginhead{Teaching experience}


\ind 2009--2013. Tutorial Assistant, teaching Metaethics, Epistemology of Perception, Descartes and Psychoanalysis. \href{http://www.heythrop.ac.uk/departments/academic-departments/philosophy/}{\emph{Heythrop College London, Department of  Philosophy} \vspace{0.01in}}

\medskip

\ind 2011--2012. Assistant Lecturer, History and Philosophy of Art, seminar leader and lecturer in Aesthetics. \href{http://www.kent.ac.uk/arts/hpa/index.html}{\emph{University of Kent, History \amper Philosophy of Art} \vspace{0.01in}}

\medskip

\ind 2011-2012. Teaching Assistant in Philosophy, seminar leader Applied Ethics. \href{http://www.ucl.ac.uk/philosophy/}{\emph{University College London, Department of  Philosophy}} \vspace{0.01in}

% \medskip
% 
% \noindent\href{http://www.hku.nl/web/English.htm}{\emph{Utrecht School of Arts} \vspace{0.01in}}
% 
% \ind 2008-2009. Assistant Lecturer in Art Theory and Cultural History.
% \medskip
% 
\ind 2007-2008. Teaching Assistant in Aesthetics. \href{http://www.uu.nl/faculty/humanities/EN/organisation/departments/departmentofphilosophy/Pages/default.aspx}{\emph{University of Utrecht, Department of  Philosophy}} \vspace{0.01in}

\ind 2008. Teaching Assistant in Moral Psychology. \href{http://www.uu.nl/faculty/humanities/EN/organisation/departments/departmentofphilosophy/Pages/default.aspx}{\emph{University of Utrecht, Department of  Philosophy}} \vspace{0.01in}

\ind 2007. Teaching Assistant and seminar leader in Philosophy of Science. \href{http://www.uu.nl/faculty/humanities/EN/organisation/departments/departmentofphilosophy/Pages/default.aspx}{\emph{University of Utrecht, Department of  Philosophy}} \vspace{0.01in}

\bigskip

\marginhead{{\vskip 0.9em}Professional \newline experience}
\medskip

\ind 2014--Present. Referee for \emph{British Journal of Aesthetics} and \emph{Topoi}.

\ind 2013--Present. Member of the Advisory Counsil, Institute of Philosophy, London.

\ind 2013--Present. Website Manager for the \href{http://www.british-aesthetics.org.uk}{British Society of Aesthetics.}

\ind 2012--2015. Editor of the \href{http:\\www.pjaesthetics.org}{\emph{Postgraduate Journal of Aesthetics.}}

\ind 2012--Present. Website editor \href{http://www.philosophy-psychoanalysis.org.uk}{London Philosophy and Psychoanalysis Group}.

\ind 2010--2015. Co-organiser \href{http:\\www.londonaestheticsforum.org}{London Aesthetics Forum}, Institute of Philosophy, London.

\ind 2009--2015. Research assistant in Philosophy at Heythrop College London. Working on philosophy of psychology, clinical methodology and psychoanalysis.

\ind 2012. Assistant organiser European Society for Philosophy and Psychology (\href{http://www.eurospp.org}{ESPP}) Annual Conference 2012, Institute of Philosophy, London. 

\ind 2007--2009. Editor of \href{http://www.cimedart.nl}{\emph{Cimedart}}, Philosophy Journal of the University of Amsterdam.

\ind 2008. Chair of the organisation for \emph{The Philosophers' Rally,} an international graduate conference in Philosophy, Utrecht.

\bigskip 

\marginhead{Professional organisations}

\ind Interdisciplinary Network on Sympathy, Empathy and Imagination (INSEI). 

\ind Non-voting member of the board of Trustees, \href{http://www.british-aesthetics.org.uk}{British Society of Aesthetics}

\ind Member, \href{http://www.philosophy-psychoanalysis.org.uk}{London Philosophy \& Psychoanalysis Group}

\ind Representative, \href{http://www.bppa-online.org}{British Postgraduate Philosophy Association} (UCL, 2011)

\bigskip

\marginhead{{\vskip 0.9em}IT-Skills}
\medskip

\ind Microsoft Office Suite; Adobe Acrobat Professional; Adobe Photoshop and Indesign; BibDesk/Endnote and TeX-based typesetting; HTML/PHP; Open Journal Systems.

\bigskip 

\marginhead{{\vskip 0.9em}Languages}
\medskip

\ind Dutch (native); English (fluent); German (good).

\rfoot{\footnotesize{\today}}

\end{document}
